\documentclass{report}

\include{preamble}
\include{macros}
\include{letterfonts}

\usepackage[Glenn]{fncychap}

\title{\Huge{Lösningar HW 13}}
\author{\huge{}}
\date{}


\begin{document}

\maketitle
\newpage
\pagebreak

\qs{}
{
\textbf{Induktion 1}: Bevisa följande:
\begin{equation}
\sum_{i=1}^{n} i^2 = \frac{n(n+1)(2n+1)}{6}
\end{equation}

\begin{equation}
\sum_{j=1}^{n} (2j-1) = n^2
\end{equation}
}
\sol (1)\\
\textbf{Basfall} $ n = 1$:
\begin{equation*}
\text{V.L:}\:\sum_{i=1}^{1} i^2 = 1;\:\:\:\text{H.L:}\: \frac{1(2)(3)}{6} = 1\:\:\:\:\text{(OK!)}
\end{equation*}
\textbf{Induktionsantagande}:
\begin{equation*}
\sum_{i=1}^{n} i^2 = \frac{n(n+1)(2n+1)}{6}
\end{equation*}
\textbf{Induktionssteget}:
\begin{equation*}
P_{V.L}(n+1) \implies \sum_{i=1}^{n+1} i^2 \iff \sum_{i=1}^{n} i^2 + (n+1)^2 \iff P_{V.L}(n) + (n+1)^2
\end{equation*}
Enligt induktionsantagandet kan vi skriva om $P_{V.L}(n)$ till $\frac{n(n+1)(2n+1)}{6}$, s.a.:
\begin{equation*}
P_{V.L}(n) + (n+1)^2 \iff \frac{n(n+1)(2n+1)}{6} + (n+1)^2	
\end{equation*}
\begin{equation*}
P_{H.L}(n+1) \implies \frac{(n+1)(n+2)(2n+3)}{6} \iff \frac{(n+1)(2n^2+7n+6)}{6} \iff \frac{(n+1)(2n^2+n+6(n+1))}{6}
\end{equation*}
\begin{equation*}
P_{H.L}(n+1) \implies \frac{(n+1)(2n^2+n+6(n+1))}{6} \iff \frac{n(n+1)(2n+1)+6(n+1)^2}{6} \iff \frac{n(n+1)(2n+1)}{6} + (n+1)^2
\end{equation*}
Det sista raden implicerar att $P_{H.L}(n+1) = P_{V.L}(n+1)$. Alltså \textbf{V.S.B}!\\\\

\sol (2)\\
\textbf{Basfall} $n=1$:
\begin{equation*}
P_{V.L}(1):\:\sum_{i=j}^{1} (2j-1) = 1;\:\:\:P_{H.L}(1):\: 1^2 = 1\:\:\:\:\text{(OK!)}
\end{equation*}
\textbf{Induktionsantagande}:
\begin{equation*}
\sum_{j=1}^{n} (2j-1)= n^2
\end{equation*}
\textbf{Induktionssteget}:
\begin{equation*}
P_{V.L}(n+1) \implies \sum_{j=1}^{n+1}(2j-1) \iff \sum_{j=1}^{n}(2j-1) + 2(n+1)-1 \iff P_{V.L}(n) + 2n+1
\end{equation*}
Enligt induktionsantagandet kan vi skriva om $P_{V.L}(n)$ till $n^2$, s.a.:
\begin{equation*}
P_{V.L}(n) + 2n+1 \iff n^2 + 2n + 1 \iff (n+1)^2 = P_{V.L}(n+1)
\end{equation*}
\begin{equation*}
P_{H.L}(n+1) \implies (n+1)^2
\end{equation*}
Alltså $P_{H.L}(n+1) = P_{V.L}(n+1)$ \textbf{V.S.B}!

\pagebreak 

\qs{}
{
\textbf{Iterativ korrekhet 2}:
}
\sol Koden kan skrivas om som en funktion:
\begin{equation*}
	f(x, n) = \prod_{i = 1}^{n} x 
\end{equation*}
Det vi måste bevisa då är att $ f(x,n) = x^n $ som kan göras med hjälp av induktionsbevis:\\\\

\noindent
\textbf{Basfall} $ n = 1 $ :
\begin{equation*}
		P_{V.L}(1): x;\:\:\:\: P_{H.L}(1): x^1\:\:\: \text{(OK!)}   
\end{equation*}
\textbf{Induktionsantagande}:
\begin{equation*}
f(x,n) = x^n
\end{equation*}
\textbf{Induktionssteget}:
\begin{equation*}
	P_{V.L}(n+1) \implies \prod_{i=1}^{n+1}x \iff x\prod_{i=1}^{n}x \iff x P_{V.L}(n)
\end{equation*}
Enligt induktionsantagandet kan vi ersätta $ P_{V.L}(n) $ till $ x^n $, s.a.:
\begin{equation*}
	P_{V.L}(n+1) \implies xP_{V.L}(n) \iff x \cdot x^n \iff x^{n+1}
\end{equation*}
\begin{equation*}
	P_{H.L}(n+1) \implies x^{n+1}
\end{equation*}
Alltså $ P_{V.L}(n+1) = P_{H.L}(n+1) $. \textbf{V.S.B}!\\\\

\noindent
Eftersom $ f(x,n) $ är implementerad med hjälp av en for-loop som i sin tur loopar i intervallen $ [1,n] $, så är tidskomplexiteten, $ O(f(x,n)) = n $ 

\qs{}
{
\textbf{Rekursiv korrekhet}:
}
\sol Eftersom vi har bevisat korrekheten för funktionen 'expIterative' så kan vi börja med att sätta basfallen till $ n = 5 $  



\end{document}
